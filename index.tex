% Options for packages loaded elsewhere
\PassOptionsToPackage{unicode}{hyperref}
\PassOptionsToPackage{hyphens}{url}
%
\documentclass[
]{article}
\usepackage{amsmath,amssymb}
\usepackage{iftex}
\ifPDFTeX
  \usepackage[T1]{fontenc}
  \usepackage[utf8]{inputenc}
  \usepackage{textcomp} % provide euro and other symbols
\else % if luatex or xetex
  \usepackage{unicode-math} % this also loads fontspec
  \defaultfontfeatures{Scale=MatchLowercase}
  \defaultfontfeatures[\rmfamily]{Ligatures=TeX,Scale=1}
\fi
\usepackage{lmodern}
\ifPDFTeX\else
  % xetex/luatex font selection
\fi
% Use upquote if available, for straight quotes in verbatim environments
\IfFileExists{upquote.sty}{\usepackage{upquote}}{}
\IfFileExists{microtype.sty}{% use microtype if available
  \usepackage[]{microtype}
  \UseMicrotypeSet[protrusion]{basicmath} % disable protrusion for tt fonts
}{}
\makeatletter
\@ifundefined{KOMAClassName}{% if non-KOMA class
  \IfFileExists{parskip.sty}{%
    \usepackage{parskip}
  }{% else
    \setlength{\parindent}{0pt}
    \setlength{\parskip}{6pt plus 2pt minus 1pt}}
}{% if KOMA class
  \KOMAoptions{parskip=half}}
\makeatother
\usepackage{xcolor}
\usepackage[margin=1in]{geometry}
\usepackage{graphicx}
\makeatletter
\def\maxwidth{\ifdim\Gin@nat@width>\linewidth\linewidth\else\Gin@nat@width\fi}
\def\maxheight{\ifdim\Gin@nat@height>\textheight\textheight\else\Gin@nat@height\fi}
\makeatother
% Scale images if necessary, so that they will not overflow the page
% margins by default, and it is still possible to overwrite the defaults
% using explicit options in \includegraphics[width, height, ...]{}
\setkeys{Gin}{width=\maxwidth,height=\maxheight,keepaspectratio}
% Set default figure placement to htbp
\makeatletter
\def\fps@figure{htbp}
\makeatother
\setlength{\emergencystretch}{3em} % prevent overfull lines
\providecommand{\tightlist}{%
  \setlength{\itemsep}{0pt}\setlength{\parskip}{0pt}}
\setcounter{secnumdepth}{-\maxdimen} % remove section numbering
\ifLuaTeX
  \usepackage{selnolig}  % disable illegal ligatures
\fi
\IfFileExists{bookmark.sty}{\usepackage{bookmark}}{\usepackage{hyperref}}
\IfFileExists{xurl.sty}{\usepackage{xurl}}{} % add URL line breaks if available
\urlstyle{same}
\hypersetup{
  pdftitle={A3: Incarceration},
  hidelinks,
  pdfcreator={LaTeX via pandoc}}

\title{A3: Incarceration}
\author{}
\date{\vspace{-2.5em}}

\begin{document}
\maketitle

\hypertarget{authors}{%
\subsubsection{Authors}\label{authors}}

Samarth Rao

\hypertarget{introduction}{%
\subsubsection{Introduction}\label{introduction}}

For this assignment, I wanted to analyze the relationship between
minorities and white people incarcerated from the late 1900s and how
this relationship and disparity changes across different locations and
city types across the U.S and also in Washington state specifically.
Even before analyzing the data, we know that in the United States, black
people have the highest rate of incarceration due to racism in our
system. However, I wanted to measure the magnitude of this data over the
years and how it can be affected across location and different
magnitudes of cities. We tend to think that many urban cities, like
those in Washington (Seattle, Bellevue) will be better then more rural
towns. Analyzing this data, can give us clues to start uncovering where
and more importantly what is causing these disparities in different
places and test our assumptions. If we understand where to look to see
these disparities, where racism could be the prevalent cause in our
problematic system, we then have a huge lead as to where to look further
to uncover more and fix the underlying issues. That is why I chose to
analyze how the location and urbanicity of counties and states can
affect the difference in blacks and white's incarcerated over the united
states

This leads us to our big question:

*\textbf{Q: How the prevalence of racism in our prison systems has
changed across differnet locations and sizes of cities in the United
States, as well as in Washington State specifically.}

\hypertarget{summary}{%
\subsubsection{Summary}\label{summary}}

To gain an understanding of our question we can look at some key
statistics that give us an insight and sneak peek of our data. Now let's
look through some of these values. We will first start off by looking at
the national datapoints. If we confine our data to the most recent date,
through some analysis we see that \textbf{LA is the state with the
highest difference} between black people and white people incarcerated,
while \textbf{KY has the smallest difference} out all U.S. states. To
look further into our question we can look specifically into
\textbf{Washington state}, to get more local values, and also explore
the impact of a county's urbanicity. Through further analysis we see
that \textbf{Cook County is the urban county with the highest
difference} of black and white people with a value of 3781.While in
rural counties, we have \textbf{East Baton Rouge Parish being the rural
county with the highest difference} with a value of 773. Obviously these
data points don't represent the whole story and are just some specific
data points, but it gives a good sneak peak at the analysis we preformed
on this dataset.

\hypertarget{the-dataset}{%
\subsubsection{The Dataset}\label{the-dataset}}

Who collected the data?

\begin{itemize}
\tightlist
\item
  The data was collected by VERA, Institute of Justice.
\end{itemize}

How was the data collected or generated?

\begin{itemize}
\tightlist
\item
  The data was collected by combining multiple sources of information
  that reports data from U.S prisons and jails. These include but are
  not limited to\ldots{} Census of Jails, FBI Uniform Crime Reporting
  Program, Center of Disease Control and more. These organizations
  report data from jails and prisons in the U.S and are combined to make
  this dataset.
\end{itemize}

Why was the data collected?

\begin{itemize}
\tightlist
\item
  There are a multitude of reasons that this data set could have been
  collected for. But the main reason to allow people to make analysis,
  and conclusions and undercover relationships and data, that be used to
  help bring justice and fairness to our incarceration system in the
  United States
\end{itemize}

How many observations (rows) are in your data?

\begin{itemize}
\tightlist
\item
  The data-set includes prison and jail data from Washington State. It
  contains 1131 rows.
\end{itemize}

How many features (columns) are in the data?

\begin{itemize}
\tightlist
\item
  There are 23 features in this data-set.
\end{itemize}

What, if any, ethical questions or questions of power do you need to
consider when working with this data?

\begin{itemize}
\tightlist
\item
  When working with this data, we need to understand that this data
  represents real people. And us handling this data, and analyzing and
  making conclusions based on this data, affects and tells a story about
  people's lives. So we need to make sure we are not falsely
  representing these people and misusing the data, and out power with
  the use of the data.
\end{itemize}

What are possible limitations or problems with this data?

\begin{itemize}
\tightlist
\item
  There are many possible limitations with this data that can affect our
  data. This dataset has many missing NA or 0 values, which can severely
  limit what we can analyze with this data, as there are data points
  that we need to include and can skew the data. Our data also has a
  problem: we need to know the reason behind these incarcerations. If we
  had a feature that showed us the reason for incarceration then we
  could make a more concrete analysis and further strengthen our claims
  by tying this in. We also have some inconsistencies in the data that
  in some of the columns make the different rates not add up to the
  total properly. This is concerning the whole population or total jail
  population. A small limitation but a limitation nonetheless is the
  fact that they only gave us the state abbreviations in the state
  feature. This made it a little harder to make maps and combine them
  with locations to analyze the data per location. Providing both would
  allow us more options and make it easier to relate to
  multiple-location data.
\end{itemize}

\hypertarget{trends-over-time-chart}{%
\subsubsection{Trends Over Time Chart}\label{trends-over-time-chart}}

\includegraphics{index_files/figure-latex/plot1-1.pdf}

\hypertarget{purpose}{%
\paragraph{Purpose}\label{purpose}}

By showing this chart we can see how the higher disproportionate rate of
blacks arrested over white has and have been changing over the years in
different city environments and populations. It lets us see the trend of
this difference over the years behaves differently in the average urban
city and the average rural cities in Washington State. From this chart
we can analyze the relationship between the red and blue lines to see
when and where the disproportionate rates of blacks incarcerated are
more common in Washington State.

\emph{Note: We went with the average to combat the fact that we had more
rural data entries than urban data entries.}

\hypertarget{insights}{%
\paragraph{Insights}\label{insights}}

\begin{itemize}
\item
  Because we defined the difference as Blacks In Jail - Whites In Jail,
  a lower number on the y-axis means that there were not that many more
  blacks in jail than whites, but a higher number tells us that there
  were significantly more blacks than whites in jails. A negative number
  means that there are more white people in jail than black people.
\item
  At a first glance of the graph we see a pretty drastic split between
  the {rural} and {urban} lines after the intial values.
\item
  The blue line which represents the difference in the amount of blacks
  minus white's incarcerated in the average urban city in Washington,
  start's to ramp up and increase, while than the red line which
  represents the same relationship in rural cities in Washington,
  decreases in the negatives after the constant line. This portrays
  after we get some values in the average Washington state, urban
  city,there is an immediate disparity between the number of black
  incarcerated opposed to whites, with more black people than white
  people arrested in urban cities.
\item
  We see the red rural line immidieatly go into the negatives with a
  downward slope and despite a few spikes up and down, never leaves the
  negatives.Our red line on the other hand has a different journey.
  After the initial start it has a few significant spikes up for
  throughout a couple years before settling back, but then settling down
  into a downward growth near the end,still staying in the positives
  though.
\item
  The lines unique behavior shows how the rate of black people
  incarcerated has been much higher in the average urban city over the
  years, while the red line seems to show that on average rural city
  there are more white people in jail then blacks. Over the years this
  number seems to have gone down for urban cities, indicating that this
  disparity is decreasing in Washington, but we still see that the
  average urban city in Washington State has a higher rate of black
  people in jail than white people.
\item
  We know from before analyzing the data that many blacks are
  incarcerated at a higher rate than whites to prevalent racism and
  through this analysis we can understand the magnitude of this
  disparity in urban cities, and how there are more white people in jail
  in rural cities, and how this disparity is more prevelant in the
  bigger urban and suburban cities of Washington State.
\end{itemize}

\hypertarget{variable-comparison-chart}{%
\subsubsection{Variable Comparison
Chart}\label{variable-comparison-chart}}

\hypertarget{insights-1}{%
\paragraph{Insights}\label{insights-1}}

\hypertarget{purpose-1}{%
\paragraph{Purpose}\label{purpose-1}}

\hypertarget{map}{%
\subsubsection{Map}\label{map}}

\includegraphics{index_files/figure-latex/heat_map-1.pdf}

\hypertarget{purpose-2}{%
\paragraph{Purpose}\label{purpose-2}}

If we remember back to the beginning of the report, our main question
that we want to answer is understanding the magnitude of this disparity
of more black people incarcerated and how it is related to the location
and urbanity of the location. This choropleth map helps us answer the
first part of this question. We can understand how the difference
changes across locations and more specifically different states. We can
see how states that are darker have a higher value for the difference
than the lighter states. This allows us to make an analysis and clearly
see where racism could be prevalent in our incarceration system.

\hypertarget{insights-2}{%
\paragraph{Insights}\label{insights-2}}

\begin{itemize}
\item
  From this chart we do see an intersting pattern with the coloration of
  the states.
\item
  We can see how the states as we go towards the east, and upper mideast
  start to get darker. The states on the west and midwest do have a few
  darker states, and semi dark states, but we see this trend more on the
  eastern side of the map.
\item
  We can see in what cluster of states near the southeast have a higher
  disparity between balcks in jail with a lot more black people in jail.
  If we wanted to see what was causing the disparity in our system then
  we could start by looking at the system in those areas.
\end{itemize}

\end{document}
